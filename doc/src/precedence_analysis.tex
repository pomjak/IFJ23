\section{Expression processing} \label{expressions}

\subsection{Syntax analysis}

%Todo:
%\begin{itemize}
%    \item describe main function - where and when is it called
%    \item precedence table
%    \item describe how syntax controls were realized
%    \item multiple errors 
%    \item multi-line expressions handling
%\end{itemize}

Whole expression processing is handled by the \verb|expression.c| module with the corresponding header file \verb|expression.h|.
The \verb|expr()| function is called whenever an expression is expected by the parser (\ref{parser:head}). This function accepts a pointer to the \verb|Parser| structure and returns the error code of first error found. 

Expressions are processed in 4 states: shift (\textless), equal shift(=), reduction(\textgreater) and error reduction(empty space).
The correct state is chosen by using precedence table \ref{table:precedence}, where the rows represent the last terminal on stack and the columns represent an incoming terminal in the current expression.

% shift and equal shift
During shift and equal shift operations a symbol is pushed on stack. In the shift state, the handle is set to the closest terminal from the top of the stack. 
In contrast with the shift state, there is no need to set the handle in equal shift state.
The handle is used also in the reduction state. 

% reduction 
Expressions are reduced by a suitable rule in reduction state. During this state  all symbols are popped from the stack (from top to handle) to an array of symbols. At most 3 symbols can be pushed to this array. These symbols are used to decide what reduction rule should be used for the operation. If there is no rule for these symbols, an error occurs.

\subsubsection{End of expression detection}
End of the expression is detected by the precedence table in error state or in the error handling function in reduction state. If the end is found in error handling functions, recovery is performed (the last pointer is set to the first symbol after the expression and the symbol is removed from from stack if necessary).


\subsubsection{Symstack data structure}
A stack implemented in the \verb|symstack.c| module is used for analyzing syntax. It contains inserted terminals and non-terminals of the current expression. This structure is a list of nodes linked from top to bottom. Every node contain data and a pointer to the previous node. The data stores information about a specific expression symbol, like a token, and basic flags that contain information about the corresponding expression symbol. The expression symbol can be either an operand, an operator or a reduced expression.  

% precedence table
\begin{table}[ht]
\centering
\label{table:precedence}
\begin{tabular}{| c | c c c | c c c | c c c |}
    \hline
        & */ & +- & ?? & i & RO &  (  &  )  & ! & \$ \\
    \hline
    */ & \textgreater & \textgreater & \textgreater & \textless & \textgreater & \textless & \textgreater & \textless & \textgreater \\
    +- & \textless & \textgreater & \textgreater  & \textless & \textgreater & \textless & \textgreater & \textless & \textgreater \\
    ?? & \textless & \textless & \textless & \textless & \textless & \textless & \textgreater & \textless & \textgreater \\
    \hline
    i  &  \textgreater & \textgreater & \textgreater &  & \textgreater &  & \textgreater & \textgreater & \textgreater \\
     RO & \textless & \textless & \textgreater & \textless & = & \textless & \textgreater & \textless & \textgreater \\
    (  & \textless & \textless & \textless & \textless & \textless & \textless & = & \textless &  \\
    \hline
    )  & \textgreater & \textgreater & \textgreater &  & \textgreater &  & \textgreater &  & \textgreater \\
    !  & \textgreater & \textgreater & \textgreater & \textgreater & \textgreater & \textgreater & \textgreater & \textgreater & \textgreater \\
    \$ & \textless & \textless & \textless & \textless & \textless & \textless & \textless &  & \textgreater \\
    \hline
\end{tabular}
\caption{Precedence table}
\begin{center}    
    \begin{minipage}{8cm}
          \small
          Note:
          \textit{i} - \textit{identifier},
          \textit{RO}- \textit{relational operator}
    \end{minipage}
\end{center}
\end{table}



\subsection{Semantic analysis}
Expression semantic analysis takes place in the reduction phase of processing the current operation. This is done across several functions with the\texttt{process\_} prefix. If possible, these functions will trigger implicit conversion of the operands. Implicit conversion is handled by \verb|convert_if_retypeable| function.

The validity of identifiers used in an expression is handled by the \verb|process_operand| function. These functions utilize the \verb|symtable| module to check whether the identifiers were previously defined.



